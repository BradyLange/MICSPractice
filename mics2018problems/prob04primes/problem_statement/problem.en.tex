\problemname{Problem 4 - Circular Primes}

Professor Plum’s doctor thinks he should lose weight, and his students think he is old. However, Professor Plum prefers to think of himself as circular prime. A circular prime number is one that remains a prime number after repeatedly relocating the first digit of the number to the end of the number.  For example, $197$, $971$, and $719$ are all cirular prime numbers. Other numbers that satisfy the circular prime definition are:  $5$, $11$, $13$, $37$, $79$, $113$, $199$, and $3119$.  \\

He wants you to write a program that finds all circular prime numbers between two given positive integers (inclusive to the numbers given). You may assume that both integers are in the range $1$ to $500000$.

\section*{Input}
A single line of input containing two integer values between $1$ and $500000$.  

\section*{Output}
The output will be an ascending list of all circular prime numbers between the two integer inputs.  The output should have one number per line.  For the example input given above, the output is: