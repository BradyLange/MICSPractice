\documentclass{article}
\usepackage[margin=0.75in]{geometry}
\pagenumbering{gobble}

\parindent 0pt

\begin{document}
%% Uncomment the next line in the final document -- it won't compile with it there though
% \problemname{Problem 3 - Invisible Ink}

%% Delete the next line in the final document -- I've included it here just to show the layout as a workaround
\section*{Problem 3 - Invisible Ink}

Professor Plum has a hard time remembering all of his passwords. He decides to store all of his passwords in a text file. To prevent someone from opening the text file and viewing his passwords, he encrypts the file using only the white-space characters of blank-spaces ('\,\,', ASCII character $32_{10}$) and horizontal-tabs ('\textbackslash t ', ASCII character $9_{10}$). Thus, someone opening the file will see only a blank screen. \\

Every $7$ space/tab characters in the file encodes a binary number where spaces represent $0$s and tabs represent $1$s. Each $7$-bit binary number encodes for an ASCII value between $0$-$127$. (NOTE: ASCII and UNICODE values are equal in this range) \\ 

Professor Plum has written the program to encrypt the passwords to a text file containing only spaces and tabs.  He wants you to write the program to decrypt this file back to the characters for the passwords.

\section*{Input}
The input contains a single line containing only a multiple of $7$ spaces and tabs, except for the ending new-line character. For example the following input (where a space is shown as ‘s’ and a tab is shown as a ‘t’) encodes the string “Hi Bob!”.  (ASCII value: ‘H’ is $72_{10}$ or $1001000_2$, ‘i’ is $105_{10}$ or $1101001_2$, ..., ‘!’ is $33_{10}$ or $0100001_2$) 

\begin{verbatim}
tsstsssttstsststssssstsssststtsttttttssstsstsssst
\end{verbatim}

\section*{Output}
The output contains only the decrypted characters corresponding to the input.  For the above example, the output would be:

\begin{verbatim}
Hi Bob!
\end{verbatim}

\end{document}