\documentclass{article}
\usepackage[margin=0.75in]{geometry}
\usepackage{graphicx}
\usepackage{amsmath}
\pagenumbering{gobble}

\parindent 0pt

\begin{document}
%% Uncomment the next line in the final document -- it won't compile with it there though
% \problemname{Problem 8 - Highly Recursive Function}

%% Delete the next line in the final document -- I've included it here just to show the layout as a workaround
\section*{Problem 8 - Highly Recursive Function}

Professor Plums likes recursion, but his students typically find it confusing.  During a recent faculty meeting his mind wandered, and he invented the following recursive mathematical function, $H(n)$:

\begin{align*}
	H(n) &= H(n+5) + H(n+4) + H(n+2)  &&\text{for all values of } n \leq -8 \\
	H(n) &= n  &&\text{for all values of } -8 <  n < 10 \\
	H(n) &= H(n-8) + H(n-5) + H(n-3) &&\text{for all values of } n \geq 10
\end{align*}

He wants you to write a program to compute values of the function $H(n)$.

\section*{Input}
The first line contains the number of $n$ values to run through the function $H(n)$.  Each of the following lines contain a single integer value of $n$.  All of the values of $n$ and corresponding $H(n)$ values will fit into a $64$-bit signed integer. The below sample input contains three $n$ values. 
\begin{verbatim}
4
-8
10
-13
-4
\end{verbatim}

\section*{Output}
For each $n$ value, print to standard output a case label and the value of $H(n)$ as defined above. For the example input given above, the output is:
\begin{verbatim}
Case 1: H(-8) = -13
Case 2: H(10) = 14
Case 3: H(-13) = -58
Case 4: H(-4) = -4
\end{verbatim}

\end{document}